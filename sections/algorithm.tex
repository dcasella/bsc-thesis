%!TEX root = ../thesis.tex

\section{Algorithm explaination}\label{section:algorithm}

The polynomial-time algorithm introduced in \cite{PPPptime2016} describes a recursive procedure for reducing a red-black graph to an empty graph, if it admits a persistent phylogeny.
The recursion stops either when the base case is reached - empty graph (line \ref{algorithm:reduce:ifempty}) - or when the algorithm can't compute a successful c-reduction for \grb{} - the graph reaches a state of irreducibility (line \ref{algorithm:reduce:ifnosource}).

The procedure for a Reduce function is given below.

\begin{algorithm}[h]
  \caption{Reduce. Recursive reduction of a red-black graph.}\label{algorithm:reduce}

  \SetKwData{Source}{\ensuremath{s}}
  \SetKwData{Sc}{\ensuremath{S_{c}}}
  \SetKwInOut{Input}{Input}
  \SetKwInOut{Output}{Output}

  \Input{Red-black graph \grb{}}
  \Output{c-reduction of the graph \grb{}, if it exists}

  \BlankLine

  Remove singletons from \grb{}\;\label{algorithm:reduce:rmsingle}

  \If{\grb{} is empty}{\label{algorithm:reduce:ifempty}
    \Return $\langle$ $\rangle$\;
  }

  \BlankLine

  \If{\grb{} has a free character \character{}}{\label{algorithm:reduce:iffree}
    \grb{} $\gets$ Realize(\character[][-], \grb{})\;

    \Return $\langle$ \character[][-], Reduce(\grb{}) $\rangle$\;
  }

  \BlankLine

  \If{\grb{} has a universal character \character{}}{\label{algorithm:reduce:ifuniversal}
    \grb{} $\gets$ Realize(\character[][+], \grb{})\;

    \Return $\langle$ \character[][+], Reduce(\grb{}) $\rangle$\;
  }

  \BlankLine

  \If{\grb{} has k > 1 connected components}{\label{algorithm:reduce:ifdisconnected}
    \Return $\langle$ Reduce($\grb{}_{0}$), \dots, Reduce($\grb{}_{k-1}$) $\rangle$\;
  }

  \BlankLine

  \gm{} $\gets$ maximal reducible graph of \grb{}\;

  \hasse{} $\gets$ Hasse diagram for \gm{}\;

  \BlankLine

  \If{\hasse{} has no safe source}{\label{algorithm:reduce:ifnosource}
    Abort\;
  }

  \BlankLine

  \Source $\gets$ Find-initial-state(\hasse{})\;

  \Sc $\gets$ sequence of positive characters of \Source that are inactive in \grb{}\;

  \grb{} $\gets$ Realize(\Sc, \grb{})\;

  \BlankLine

  \Return $\langle$ \Sc, Reduce(\grb{}) $\rangle$\;
\end{algorithm}

The Reduce procedure computes signed character realizations with the support of a Realize function, which follows the definition of Realization (\ref{definition:realization}) quite literally. \fix{Reword} We still describe the procedure for the Realize function.

\pagebreak % temporary - delete

\begin{algorithm}[h]
  \caption{Realize. Realization of a list of signed characters in a red-black graph.}\label{algorithm:realize}

  \SetKwData{Lc}{\ensuremath{L_{c}}}
  \SetKwData{Con}{\ensuremath{Conn(\character{})}}
  \SetKwData{Adj}{\ensuremath{S(\character{})}}
  \SetKwInOut{Input}{Input}
  \SetKwInOut{Output}{Output}

  \Input{List of signed characters \Lc of \grb{}}
  \Input{Red-black graph \grb{}}
  \Output{Red-black graph \grb{} after the realization of \Lc, if feasible}

  \BlankLine

  \ForEach{\character[][\pm] $\in$ \Lc}{
    \Con $\gets$ species in the connected component that contains \character{}\;
    \Adj $\gets$ species adjacent to \character{}\;

    \BlankLine

    \If{\character[][+] is inactive}{
      Add red edges between \character{} and each species in $\Con \setminus \Adj$\;

      Delete black edges incident on \character{}\;
    }
    \ElseIf{\character[][-] is active}{
      Delete edges incident on \character{}\;
    }
    \Else{
      Abort\;
    }
  }

  \BlankLine

  \Return \grb{}\;
\end{algorithm}

Notice that a realization may alter the connected components of the red-black graph; we then need to recompute (or update) the connected component of a character \character[][\pm] before its realization.

To ensure that a character external to $L_{c}$ gets realized whenever possible, the procedure can be extended by adding checks for free and universal characters after each realization.

Finally, a step that removes singletons from the red-black graph can be implemented right before the \textbf{return} statement.

We will address the Find-initial-state procedure in detail in section \ref{section:safe-chains-sources}.

\subsection{Preparing the graph}\label{section:preparing-the-graph}

An explaination of lines \ref{algorithm:reduce:rmsingle} to \ref{algorithm:reduce:ifdisconnected} has already been provided in section \ref{section:grb}.

The first steps of the Reduce procedure (algorithm \ref{algorithm:reduce}) serve the purpose of bringing the red-black graph to a state which can't be "pruned" further.
Removing isolated vertices (singletons), realizing free and universal characters, and reducing each connected component of \grb{} separately is all part of preparing the graph for a more thorough analysis.

\subsection{Maximal reducible graph and Hasse diagram}\label{section:gm-hassediagram}

\subsection{Safe chains and sources}\label{section:safe-chains-sources}
