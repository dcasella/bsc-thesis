%!TEX root = ../thesis.tex

\section{Algorithm explaination}\label{section:algorithm}

The polynomial-time algorithm introduced in \cite{PPPptime2016} describes a recursive procedure for reducing a red-black graph to an empty graph, if it admits a persistent phylogeny.
The recursion can be stopped in one of two ways: by reaching an empty graph (\hyperref[algorithm:reduce:ifempty]{line 2}) or by reaching a graph that can't be reduced (\hyperref[algorithm:reduce:ifnosource]{line 14}).

The procedure for a Reduce function is given below.

\begin{algorithm}[h]\label{algorithm:reduce}
  \caption{Reduce. Recursive reduction of a red-black graph.}

  \SetKwData{Source}{\ensuremath{s}}
  \SetKwData{Sc}{\ensuremath{S_{c}}}
  \SetKwInOut{Input}{Input}
  \SetKwInOut{Output}{Output}

  \Input{Red-black graph \grb{}}
  \Output{c-reduction of the graph \grb{}, if it exists}

  \BlankLine

  Remove singletons from \grb{}

  \If{\grb{} is empty}{\label{algorithm:reduce:ifempty}
    \Return $\langle$ $\rangle$
  }

  \BlankLine

  \If{\grb{} has a free character \character{}}{
    \grb{} $\gets$ Realize(\character[][-], \grb{})

    \Return $\langle$ \character[][-], Reduce(\grb{}) $\rangle$
  }

  \BlankLine

  \If{\grb{} has a universal character \character{}}{
    \grb{} $\gets$ Realize(\character[][+], \grb{})

    \Return $\langle$ \character[][+], Reduce(\grb{}) $\rangle$
  }

  \BlankLine

  \If{\grb{} has k > 1 connected components}{
    \Return $\langle$ Reduce($\grb{}_{0}$), \dots, Reduce($\grb{}_{k-1}$) $\rangle$
  }

  \BlankLine

  \gm{} $\gets$ maximal reducible graph of \grb{}

  \hasse{} $\gets$ Hasse diagram for \gm{}

  \BlankLine

  \If{\hasse{} has no safe source}{\label{algorithm:reduce:ifnosource}
    Abort
  }

  \BlankLine

  \Source $\gets$ Find-initial-state(\hasse{})

  \Sc $\gets$ sequence of positive characters of \Source that are inactive in \grb{}

  \grb{} $\gets$ Realize(\Sc, \grb{})

  \BlankLine

  \Return $\langle$ \Sc, Reduce(\grb{}) $\rangle$
\end{algorithm}

\todo[inline]{Add other functions' pseudocode?}

\subsection{Preparing the graph}\label{section:preparing-the-graph}

\subsection{Maximal reducible graph and Hasse diagram}\label{section:gm-hassediagram}

\subsection{Safe chains and sources}\label{section:safe-chains-sources}
